\section{Theorie und Vorwissen}
\subsection{Zephyr}
    Zephyr ist ein Open-Source-Echtzeitbetriebssystem welches von der Linux Foundation.\footnote{Quelle: \url{https://de.wikipedia.org/wiki/Zephyr_(Betriebssystem)}}
    Ein Echtzeitbetriebssystem, real-time operating system \textbf{RTOS} ist ein Betriebssystem, das Echtzeit-Anforderungen erfüllen kann. 
    Das bedeutet, dass Anfragen eines Anwendungsprogramms innerhalb einer Voraus bestimmbaren Zeit gesichert verarbeitet werden.\footnote{Quelle: \url{https://de.wikipedia.org/wiki/Echtzeitbetriebssystem}}
    \\
    Zephyr wurde mit dem \href{https://docs.zephyrproject.org/latest/getting_started/index.html}{Getting-Started-GUID} Linux Subsystem von Windows installiert. 
    Um ein Zephyr Projekt zu kompilieren wird Zephyr eigenes \textbf{West}\footnote{\url{https://docs.zephyrproject.org/2.4.0/guides/west/index.html}} verwendet.\\
    \textbf{West} ist ein Kompilierungs-Tool von Zephyr. Es verwendet Ninja und CMake um das Projekt zu kompilieren. 
    West wird folgendermaßen verwendet, um ein Projekt zu kompilieren: 
    \begin{lstlisting}[style=StyleC, captionpos=b, caption=West Beispiel, label=West Beispiel]
west build -p auto -b nativ_posix_64 
    \end{lstlisting}

    \subsubsection{KConfig}
    \textbf{Kernel Configuration File}\footnote{\url{https://docs.zephyrproject.org/latest/application/index.html?\#application-kconfig}}ist die \textcolor{red}{prj.conf} Datei in einem 
    Zephyr Projekt. In diesem werden bestimmte Konfigurationen, Funktionen und \anfuehrung{Geräte}, wie z.b. \textit{CONFIG\_SERIAL=y} aktiviert. 


\newpage
    \subsubsection{Device Tree}
    Der Device Tree\footnote{\url{https://docs.zephyrproject.org/latest/guides/dts/intro.html}\\\url{https://docs.zephyrproject.org/latest/reference/devicetree/index.html\#devicetree}} ist 
    in einem Zephyr Projekt eine Datei mit der Endung \textbf{.dts} dort stehen alle für das ausgewählte Board verfügbare Geräte drinnen.
    Im Fall des nativ\_posix\_64 sieht dieses folgendermaßen aus. 
    \begin{lstlisting}[style=StyleC, captionpos=b, caption=West Beispiel, label=West Beispiel]
/dts-v1/;

/ {
    #address-cells = < 0x1 >;
    #size-cells = < 0x1 >;
    model = "Native POSIX Board";
    compatible = "zephyr,posix";
    chosen {
        zephyr,console = &uart0;
        zephyr,shell-uart = &uart0;
        zephyr,uart-mcumgr = &uart0;
        zephyr,flash = &flash0;
        zephyr,entropy = &rng;
        zephyr,flash-controller = &flashcontroller0;
        zephyr,ec-host-interface = &hcp;
    };
    aliases {
        eeprom-0 = &eeprom0;
        i2c-0 = &i2c0;
        spi-0 = &spi0;
        led0 = &led0;
    };
    leds {
        compatible = "gpio-leds";
        led0: led_0 {
            gpios = < &gpio0 0x0 0x0 >;
            label = "Green LED";
        };
    };

    ...

    };
    uart0: uart {
        status = "okay";
        compatible = "zephyr,native-posix-uart";
        label = "UART_0";
        current-speed = < 0x0 >;
    };
    
    ...

};    
    \end{lstlisting}


\newpage
    \subsubsection{Tinycrypt}
\subsection{Linux Pseudo-Terminal}
\subsection{Threads}
\subsection{Message-Queue}
