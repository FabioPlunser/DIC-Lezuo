\newpage
\section{Aufgabenstellung}
    Die Aufgabe ist einfach (formuliert), implementieren Sie in Zephyr RTOS einen Krypto Prozessor der allen in test.py durchgeführten
    Tests erfolgreich absolviert.
    Die mitgelieferte Datei zephyr.elf st eine Referenezimplementierung (unter 64 Bit Ubuntu 20.04, zum Beispiel auch unter Windows
    10 / WSL lauffähig). Eine korrekte Lösung verhält sich wie in diesem Dokument beschrieben.
    

\subsection{Aufgaben und Eigenschaften des Krypto Prozessors}
    Der Kyprto Prozessor kann über eine serielle Schnittstelle angesprochen werden und führt für einen Benutzer AES-128 Operationen
    durch. Es wird das RTOS Zephyr in Version 2.4.0 als Basis verwendet. Diese Code ist damit prinzipiell auf einer Reihe von
    Microcontroller lauffähig. Wir verwenden als "Microcontroller Board" ein 64bit Linux (board \textbf{native\_posix\_64} in Zephyr Sprech). Da
    wird für die Linux Entwicklung keine SDKs benötigen muss dieses auch nicht installiert werden. Geben Sie beim Erzeugen der
    Buildumgebung daher bitte (\textbf{CROSS\_COMPILE}= und \textbf{ZEPHYR\_TOOLCHAIN\_VARIANT=cross-compile}) an.
    \\
    \\
    Das native\_posix\_64 Board implementiert eine virtuelle serielle Schnittstelle namens \textbf{UART\_0}, sowie eine implementierung der
    crypo API mittels libtinycrypt \textbf{CRYPTO\_TC}. Diese beiden Treiber sind zu verwenden.
